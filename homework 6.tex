\documentclass[12pt,twoside]{article}
\usepackage[dvipsnames]{xcolor}
\usepackage{tikz,graphicx,amsmath,amsfonts,amscd,amssymb,bm,cite,epsfig,epsf,url}
\usepackage[hang,flushmargin]{footmisc}
\usepackage[colorlinks=true,urlcolor=blue,citecolor=blue]{hyperref}
\usepackage{amsthm,multirow,wasysym,appendix}
\usepackage{array,subcaption} 
% \usepackage[small,bf]{caption}
\newcommand{\red}[1]{{\leavevmode\color{red}{#1}}}
\newcommand{\blue}[1]{{\leavevmode\color{blue}{#1}}}
\usepackage{enumitem}


\makeatletter
\renewcommand*\env@matrix[1][*\c@MaxMatrixCols c]{%
  \hskip -\arraycolsep
  \let\@ifnextchar\new@ifnextchar
  \array{#1}}
\makeatother

\begin{document}

\begin{center}
{\large{\textbf{Homework 6}} } \vspace{0.2cm}\\
Due October 16thth at 12 am
\\
\end{center}
\begin{enumerate}[label=6.1]
    \item compute all eigenvalues of \begin{pmatrix}0&1\\ 1&0\end{pmatrix} and asociated eigenvectors

\blue{
\begin{itemize}
    \item we know that the eigenvalues of a matrix are the values such that $Ax=\lambda x$ meaning that $Ax-\lambda x=0=x(Ax-I \lambda)$ this condition is equivalent finding a $\lambda \in \mathbb{R}$ such that $det(AX-I\lambda)=0$ 
    \item so we can write $Ax-\lambda I=$ \begin{pmatrix} 0&1-\lambda\\ 1-\lambda & 0 
    \end{pmatrix}
    \item and see $det(\begin{pmatrix} -\lambda&1\\ 1 & -\lambda 
    \end{pmatrix})=1-(-\lambda)(-\lambda)^{2}=1-\lambda^2$ this equals zero when $\lambda=1\text{ or } \lambda=-1$
    \item then we want vectors such that $A(-\lambda I) x=0$ meaning that for $\lambda=1$ we want $\begin{pmatrix}-1&1\\ 1&-1\end{pmatrix}x=0$ adding row 1 to row 2 we get $\begin{pmatrix}-1&1\\ 0&0\end{pmatrix}x=0$ and deviding row 1 by -1 yields $\begin{pmatrix}1&-1\\ 0&0\end{pmatrix}x=0$
    \item this implies that $x_1-x_2=0$ ie that $x_1=x_2$ for eigenvalue 1 is $\begin{pmatrix} 1&1 \end{pmatrix}$
    \item repeating the same process for for $\lambda=-1$ we want $\begin{pmatrix}-1&-1\\ 1&1\end{pmatrix}x=0$ adding row 1 to row 2 we get $\begin{pmatrix}-1&-1\\ 0&0\end{pmatrix}x=0$ and dividing row 1 by -1 yields $\begin{pmatrix}1&1\\ 0&0\end{pmatrix}x=0$
    \item this implies that $x_1+x_2=0$ ie that $x_1=-x_2$ thus a simple eigenvector for eigenvaue -1 is $\begin{pmatrix} 1&-1 \end{pmatrix}$
\end{itemize}
    }
\end{enumerate}    
    



\begin{enumerate}[label=6.2]
\item Let v $\in \mathbb{R}^N$ be a non-zero vector. What are the eigenvalues of the n × n
matrix $M = vv^T$ ? What are the multiplicities of these eigenvalues? Justify
\blue{
\begin{itemize}
    \item for any  v $\in \mathbb{R}^N$ we can see that $M=v*v^t=$\begin{pmatrix}v_1^2&v_1v_2&...&v_1v_n\\...&...&...&...\\v_1v_n&v_nv_2&...&v_n^2\end{pmatrix}=\begin{pmatrix}v_1(v)\\...\\v_n(v)\end{pmatrix}writing it in this way makes it clear that each row is really just a linear combination of one another. 
    \item thus we can see that $Rank(M)=1$ this implies that $dim(ker(m))=n-1$
    \item as the kernel is not empty we know there exists at least 1 vector $x\in \mathbb{R^n}$ such that $Ax=0(x)=0$ meaning that zero is indeed an eigenvalue
    \item further as we know that the multiplicity of a certain eigenvalue is $dim(ker(m-\lambda i))$ so we can see that $E_{0}(M)=dim(ker(m-0 i))=dim(ker(M))=n-1$
    \item additionally we know that over the k potential eigenvalues with multiplicity n $\Sigma_{i}^{k}m_i\leq n$  meaning that there can at most be one more eigenvalue of multiplicity one. 
    \item note that $Mv\neq 0$ meaning that $x\not \in ker(M)$ and thus $x\in span(v)$ meaning that the reaming eigenvector x is in the span(v) so we can just find the eigenvalue for v
    in other words any $\lambda \in \mathbb{R}: Mv=\lambda v$ must be the final remaining eigenvalue. \item so we can write $Mv=\lambda v$
    \item $Mv-\lambda v=0=(M-\lambda i)v=0$
    \item $((M-\lambda I)=v*v^t-\lambda I)v=0=$\begin{pmatrix}(v_1^2-\lambda  i)v_1 &v_1v_2v_2&...&v_1v_nv_n\\...&...&...&...\\v_1v_nv_1&v_nv_2&...&(v_n^2\lambda i)v_n\end{pmatrix}=\\\begin{pmatrix}(v_1(v_1^2+v_2^2+...+v_n^2-\lambda)  \\...\\(v_n(v_1^2+v_2^2+...+v_n^2-\lambda)\end{pmatrix}
\\\rightarrow \lambda =\Sigma_{i=1}^{n}v_i^2

\item so finally we see we have two eigenvalues $\lambda_0=0$ and $\lambda_1=\Sigma_{i=1}^{n}v_i^2$
and $E_{0}=n-1$ and $E_{\Sigma_{i=1}^{n}v_i^2}=1$
\end{itemize}
}
\end{enumerate}

\begin{enumerate}[label=6.3]
\item Consider a Washington Square squirrel trapped in a box divided into
9 rooms. This squirrel enjoys an active life style, so every minute, it decides to go through any of
the available doors or stay in the current room, where all actions are taken with equal probability.
See figure. For example, if the squirrel is in box 1 at time t, then at time t + 1, it is in box 1, 2,
or 4 with probability 1/3 each. For every minute t = 0, 1, 2, . . . , let x(t) ∈ R9 denote the vector
whose i-th entry is the probability that the squirrel is in box i at time t
\begin{enumerate}
    \item Find a matrix P such that x(t + 1) = P x(t)
    \blue{
    \begin{itemize}
        \item We know that there are 9 total states, and at each state there is a probability that the squirrel can transition to any other state
        \item thus we have a 9X9 matrix 
        \item where each entree  i is the likelihood of ending up in sate i starting at state j. that is $P_{i,j}=p(i|j)$
        \ITME all of this is with markov assumptions by the way 
        \item so this gives us the following P matrix \\$P=\begin{pmatrix}
        1/3 & 1/4 & 0 & 1/4 & 0 & 0 & 0 & 0 & 0  
        \\1/3 & 1/4 & 1/3 & 0 & 1/5 & 0 & 0 & 0 & 0 \\0 & 1/4 & 1/3 & 0 & 0 & 1/4 & 0 & 0 & 0 \\1/3 & 0 & 0 & 1/4 & 1/5 & 0 & 1/3 & 0 & 0 \\0 & 1/4 & 0 & 1/4 & 1/5 & 1/4 & 0 & 1/4 & 0 \\0 & 0 & 1/3 & 0 & 1/5 & 1/4 & 0 & 0 & 1/3
        \\0 & 0 & 0 & 1/4 & 0 & 0 & 1/3 & 1/4 & 0 
        \\0 & 0 & 0 & 0 & 1/5 & 0 & 1/3 & 1/4 & 1/3 
        \\0 & 0 & 0 & 0 & 0 & 1/4 & 0 & 1/4 & 1/3 \\
        \end{pmatrix}$
    \end{itemize}
    
    }
    \item Can you find an integer k ≥ 1 such that all entries of $P^k$ are strictly
positive?
\blue{
\begin{itemize}
    \item we can computationally check that $P^{k}$ holds for k=4
    
\end{itemize}

}
\item Suppose that at time 0, the squirrel is in room 1. In 10 minutes,
what is the probability the squirrel will be in each of the 9 rooms?
(That is, if x(0) = e1 ∈ R9, then compute x(10)
\item compute the invariant measure for this 
\item You plan to visit NYU in 10 years to see your friend, this squirrel.
Which room should you arrive in so as to maximize the probability
that when you arrive, the squirrel is in the same room? Justify
\end{enumerate}
\end{enumerate}

\end{document}
