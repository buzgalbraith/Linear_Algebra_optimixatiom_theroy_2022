\documentclass[12pt,twoside]{article}
\usepackage[dvipsnames]{xcolor}
\usepackage{tikz,graphicx,amsmath,amsfonts,amscd,amssymb,bm,cite,epsfig,epsf,url}
\usepackage[hang,flushmargin]{footmisc}
\usepackage[colorlinks=true,urlcolor=blue,citecolor=blue]{hyperref}
\usepackage{amsthm,multirow,wasysym,appendix}
\usepackage{array,subcaption} 
% \usepackage[small,bf]{caption}
\newcommand{\red}[1]{{\leavevmode\color{red}{#1}}}
\newcommand{\blue}[1]{{\leavevmode\color{blue}{#1}}}
\usepackage{enumitem}


\makeatletter
\renewcommand*\env@matrix[1][*\c@MaxMatrixCols c]{%
  \hskip -\arraycolsep
  \let\@ifnextchar\new@ifnextchar
  \array{#1}}
\makeatother

\begin{document}

\begin{center}
{\large{\textbf{Homework 9}} } \vspace{0.2cm}\\
Due October 16thth at 12 am
\\
\end{center}
\begin{enumerate}[label=9.1]
\item  are the following functions convex
\begin{enumerate}
    \item $f(x)=1024x-2022$
    \begin{itemize}
       \item a function being convex is equivalent to it's Hessian matrix being Positive semi definite. \item as this is a function in $\mathbb{r}$ we know that its gradient is just it's darivative that is $\nabla_f(x)=$\begin{pmatrix} 1024\end{pmatrix}
       \item and further it has Hessian matrix $\mathcal{H}_f(x)=$\begin{pmatrix} 0\end{pmatrix} for any values of x
       \item we know that a matrix of demsion $1X1$ can have at most 1 eigen vectors and further it is celear that for any $y\in \mathbb{R}$ $\mathcal{H}_f(x)y=$\begin{pmatrix} 0\end{pmatrix}=0y
       \item thus we can see that 0 is the only eigen value of the hessian matrix of f meaning it is PSD and thus f is convex
    \end{itemize}
    \item $f(x)=x^{11}$
        \begin{itemize}
    \item f(x) is not convex. 
    \item we can see that $f'(x)=11x^10$ and $f''(x)=120(x)^{9}$ 
    \item if we evaluate $f''(x)=120(x)^{9}$ at x=-1 then we can see that $f''(-1)=-120$
    \item and clearly a matrix with its only value being -120 will have it's only eigne vlaue as -120, meanign that the hessian of f is not always postive semi defininte and thus f can not be covex.
    \end{itemize}
    \item $f(x)=x^{2022}$
        \begin{itemize}
        \item now if we want to consider $f(x)=x^{2022}$
        \item we can compute it's hesian matrix as $\mathcal{H}_{f}(x)=2022(2021)(x)^{2020}$ we can see that $x^{2020}\geq $0 as the value is raised to an even power and $2022(2021)\geq 0$ meaning that $\mathcal{H}_{f}(x)=2022(2021)(x)^{2020}\geq 0$ for all values of x and thus the function is convex.  
    \end{itemize}
    \item $f(x)=e^x$
        \begin{itemize}
        \item here we can see that $\mathcal{H}_f(x)=\begin{pmatrix}e^x \end{pmatrix}$ further it is known that $e^x\geq0$ for all values of x, thus f is convex.
    \end{itemize}
    
\end{enumerate}
\newpage
\end{enumerate}
\begin{enumerate}[label=9.2]
\item let $M \in\mathbb{R}^{n×n}$
be a symmetric matrix, $b\in\mathbb{R}^{n}$
, and $c\in\mathbb{R}$. Consider
the quadratic function f : $\mathbb{R}$
n $\rightarrow \mathbb{R}$ given by
$f(x) = x^t
Mx + <x,b> + c$.
\begin{enumerate}
    \item compute the gradient of f. 
    \begin{itemize}
        \item note that $f(x)$ can be expressed as $f(x)=\Sigma_{i=1}^{n}\Sigma_{j=1}^{n}m_{i,j}x_{i}x{j}+\Sigma_{k=1}^{n}x_kb_k$
        \item differentiating this with respect to x yields $\nabla_f(x)=\begin{pmatrix}\Sigma_{i=1}^{n}m_{i,1}x_i+\Sigma_{k=1}^{n}m_{1,j}x_j+b_1\\...\\\Sigma_{i=1}^{n}m_{i,n}x_i+\Sigma_{k=1}^{n}M_{n,j}x_j+b_n 
        \end{pmatrix}$ further because M is symmetric we have $\nabla_f(x)=\begin{pmatrix}2\Sigma_{i=1}^{n}m_{i,1}x_i+b_1\\...\\2\Sigma_{i=1}^{n}m_{i,n}x_i+b_n 
        \end{pmatrix}=2Mx+b$
    \end{itemize}
    \item compute the hessian of f 
        \begin{itemize}
        \item from part a we can write the gradient vector of f(x) as  $\nabla_f(x)=\begin{pmatrix}2\Sigma_{i=1}^{n}m_{i,1}x_i+b_1\\...\\2\Sigma_{i=1}^{n}m_{i,n}x_i+b_n 
        \end{pmatrix}$ 
        \item differentiating this with respect to x yields. $\mathcal{H}_{f}(x)=2M$
    \end{itemize}
    \item  Use part (b) to prove that f is convex if and only if M is positive semidefinite
        \begin{itemize}
        \item 2 is just a scalar so the eigenvalues of the Hessian of f 2M will have the same sign as those of M. 
        \item further we know that 2M is the hessian, so if M is PSD 2M is PDS and thus the hesisan matrix of f is PSD which is equivalent to the function f being convex. 
    \end{itemize}
    \item Use part (c) to prove that the function $g:\mathbb{R}^2
 \rightarrow \mathbb{R}$ given by $g(x) =x_1^2-x_2^2$
is not convex.
    \begin{itemize}
        \item we can see that $\nabla_{g}(x)=\begin{pmatrix} 2x_1\\-2x_2\end{pmatrix}$ 
        \item from this we find that $\mathcal{H}_{g}(x)=\begin{pmatrix}2&0\\0&-2\end{pmatrix}$ for any value of x. 
        \item we can see that that $\mathcal{H}_{g}(x)\begin{pmatrix}0\\1\end{pmatrix}=\begin{pmatrix}2&0\\0&-2\end{pmatrix}\begin{pmatrix}0\\1\end{pmatrix}=\begin{pmatrix}0\\-2\end{pmatrix}=-2\begin{pmatrix}0\\1\end{pmatrix}$ 
        \item thus -2 is an eigenvalue of the hessian matrix of H, and there for the function g(x) is not convex.
    \end{itemize}
\end{enumerate}


\newpage
\begin{enumerate}[label=9.3]
\item  Let$A\in \mathbb{R}^{nXm} $and$y\in\mathbb{R}^n$  For $x \in \mathbb{R}^m$
 we define
$f(x) = ||  Ax - y||^2$
\end{enumerate}
\begin{enumerate}
    \item compute the gradient and hessian matrix
    \begin{itemize}
        \item notice that we can express $f(x) = ||  Ax - y||^2=<Ax-y,Ax-y>=(Ax-y)^{t}(Ax-y)=(x^tA^t-y^t)(Ax-y)=x^tA^tAx-x^tA^ty-y^tAx+y^ty=x^tA^tAx-<x,A^ty>-<A^ty,x>+<y,y>=x^tA^tAx-<x,A^ty>-<x,A^ty>+<y,y>$
        \item we can define $M\in \mathbb{R}^{MxM}$ such that $M=A^{t}A$
        \item we can define $b\in \mathbb{R}^{M}$ such that $b=A^ty$
        \item and we can define $c\in \mathbb{R}$ such that $c=<y,y>$
        \item this finally allows us to write $f(x) = ||  Ax - y||^2=x^tMx+<x,b>+c$
        \item thus we find $\nabla_{f}(x)=2Mx+b=2(A^{t}A)x-<2A,y>$        \item and $\mathcal{H}_{f}(x)=2M=2(A^{t}A)$
    \end{itemize}
\item show that f is convex.
\begin{itemize}
    \item we know that hessian matrix of A is equal to $2M=2(A^tA)$.
    \item as we argued earler, the eigen vectors of $M$ are just the eigenvectors of $2m$ scalled by a factor of $\frac{1}{2}$ so we know that the sign of all the eigen vectors will be the same. 
    \item further as  we demonstrated in homework 7 question 7.2.D that for any matrix A we can deffine another matrix $M=A^tA$ whcih is is positive semi definite. 
    \item thus we know that $A^tA=M$ is postive semi defininte and thus that the hessian matrix of f $\mathcal{H}_{f}(x)=2M=2A^tA$ is also postive semi definite which is equivlent to F being convex
    \item thus f is convex.
\end{itemize}
\end{enumerate}
\newpage
\end{enumerate}
\begin{enumerate}[label=9.4]\item let $f:\mathbb{R}^{n}\rightarrow \mathbb{R}$ be a convex function. Assume that the minimum
$m = min_{x\in\mathbb{R}^n}f(x)$ of f on R
n
is finite, and that the set of minimizes of f
$M=\{v\in\mathbb{R}^{n}|(v) = m\}$
is non-empty.
\end{enumerate}
\begin{enumerate}
\begin{enumerate}
    \item show that M is a convex set
    \begin{itemize}
        \item consider $x,y\in M: f(x)=f(y)=m$
        \item we know as f is convex $f(\alpha x+ (1-\alpha)y)\leq \alpha f(X)+(1-\alpha) f(y)=\alpha m +(1-\alpha) m = m(1-\alpha+\alpha)$ this means that for any value of $x,y$ we have $\alpha x+ (1-\alpha)y\in M$ meaning the set M is convex.   
    \end{itemize}
    \item show that if f is strictly convex then M has only one element. 
    \begin{itemize}
    \item suppose for the sake of contradicion $x,y\in M: x\neq y$ and f is striclty convex 
    as f is convex we have $f(\alpha x+ (1-\alpha)y)\leq \alpha f(X)+(1-\alpha) f(y)$ but as $x,y\in M$ we have $\alpha m +(1-\alpha) m = m(1-\alpha+\alpha)=m$
    \item then as $x\neq y$ and $f(x)\alpha+(a-\alpha)f(y)=m$ f can not be strictly convex. whcih is a contradiction
    
\end{itemize}
\end{enumerate}
\end{enumerate}
\end{document}


