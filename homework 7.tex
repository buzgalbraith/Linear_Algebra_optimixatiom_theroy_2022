\documentclass[12pt,twoside]{article}
\usepackage[dvipsnames]{xcolor}
\usepackage{tikz,graphicx,amsmath,amsfonts,amscd,amssymb,bm,cite,epsfig,epsf,url}
\usepackage[hang,flushmargin]{footmisc}
\usepackage[colorlinks=true,urlcolor=blue,citecolor=blue]{hyperref}
\usepackage{amsthm,multirow,wasysym,appendix}
\usepackage{array,subcaption} 
% \usepackage[small,bf]{caption}
\newcommand{\red}[1]{{\leavevmode\color{red}{#1}}}
\newcommand{\blue}[1]{{\leavevmode\color{blue}{#1}}}
\usepackage{enumitem}


\makeatletter
\renewcommand*\env@matrix[1][*\c@MaxMatrixCols c]{%
  \hskip -\arraycolsep
  \let\@ifnextchar\new@ifnextchar
  \array{#1}}
\makeatother

\begin{document}

\begin{center}
{\large{\textbf{Homework 7}} } \vspace{0.2cm}\\
Due October 16thth at 12 am
\\
\end{center}
\begin{enumerate}[label=7.1]
   \item  Let $A \in \mathbb{R}^{nxx}$
   \begin{enumerate}
       \item Show for $B\in\mathbb{R}^{nxm}$ and $C\in\mathbb{R}^{mxn}$ we have $tr(BC)=TR(CB)$
       
       \red{
       \begin{itemize}
           \item call $D=CB$, $E=BC$
           \item note that any ellement of D can be expressed as $D_{i,j}=\Sigma_{k=1}^{n}C_{i,k}B_{k,j}$
           \item while any ellement of E can be expressed as $E_{i,j}=\Sigma_{k=1}^{m}B_{i,k}C_{k,j}$
           \item so if we take the sum of the ellements on the diagnol of D we see $\Sigma_{i=1}^{m}D_{i,i}=\Sigma_{i=1}^{m}\Sigma_{k=1}^{n}C_{i,k}B_{k,i}$
           \item while the diagnaol ellementts of E are  $\Sigma_{k=1}^{n}E_{k,k}=\Sigma_{i=1}^{n}\Sigma_{i=1}^{n}B_{k,i}C_{i,k}$ then by linarty of sums we have that equals $\Sigma_{i=1}^{m}\Sigma_{k=1}^{n}C_{i,k}B_{k,i}$ proving the equality
       \end{itemize}
       }
      \item prove that for any symetric matrix A the trace of A is equal to the sum of its eigenvalues. 
      \red{
      \begin{itemize}
          \item notice first that as A is symmetric and square the spectral theorem applies meaning A can be expressed as $A=PDP^{T}$ meaning that  $TR(A)=TR(PDP^T)$ then we know that then by part A we can say that $TR(A)=TR(PDP^T)=TR(P^TDP)$
          \item so notice that we can write $P^TDP=$\begin{pmatrix} V_1^{T}\\...\\V_n^{T}
          \end{pmatrix}\begin{pmatrix} \lambda_1 & ...& 0 \\...& ..... &...\\0 &... & \lambda_n 
          \end{pmatrix}
          \begin{pmatrix} V_1&..&V_n^{T}
          \end{pmatrix}=\begin{pmatrix} v_{1,1}^t\lambda_1 & v_{1,2}^t\lambda_2..& v_{1,n}^t\lambda_n \\...& ..... &...\\v_{n,1}^t\lambda_1 & v_{n,2}^t\lambda_2..& v_{n,n}^t\lambda_n  \lambda_n 
          \end{pmatrix}
          \begin{pmatrix} V_1&..&V_n^{T}
          \end{pmatrix}=\\\begin{pmatrix}
          <\lambda_1 v_1,& v_1>&<\lambda_2 v_2, v_1>&...&<\lambda_n v_n, v_1>\\...&...&...\\<\lambda_1 v_1,& v_n>&<\lambda_2 v_n, v_1>&...&<\lambda_n v_n, v_n>
          \end{pmatrix}=\\\begin{pmatrix}
          \lambda_1< v_1,& v_1>&\lambda_2< v_2, v_1>&...&\lambda_n< v_n, v_1>\\...&...&...\\\lambda_1< v_1,& v_n>&\lambda_2< v_n, v_1>&...&\lambda_n< v_n, v_n>\end{pmatrix}=\begin{pmatrix} \lambda_1 & ...& 0 \\...& ..... &...\\0 &... & \lambda_n 
          \end{pmatrix} 
          \item and thus the trace is the sum of the our lambdas
      \end{itemize}
      }
   \end{enumerate}
\end{enumerate}
\begin{enumerate}[label=7.2]
\item
\begin{enumerate}
    \item let a be a symmetric matrix show that A is positive semidefninte only if  for any $y\in \mathbb{R}^{n} x^TDx\geq0$
    \red{
    \begin{enumerate}
        \item first we will show that if A is positive semi-definite then for any $y\in \mathbb{R}^{n} y^TDy\geq0$
        \begin{itemize}
            \item consider an abitarty vector $x\in \mathbb{R}^n$ first notice that $x^tV=\begin{pmatrix}
            x_1&...&x_n
            \end{pmatrix}\begin{pmatrix}
            v_1&...&v_n
            \end{pmatrix}=\begin{pmatrix}
            y_1(v_1)&...&y_n(v_n)
            \end{pmatrix}$ where $y_i$ are scalars and $v_i$ are vectors. ie that there product is a linear combination of the bassic vectors. 
            \item so since we know $\mathbb{R}^{n} y^TAy\geq0$ for any value of y, we can manipulate our value of y such that we can have $\begin{pmatrix}
            y_1(v_1)&...&y_n(v_n)
            \end{pmatrix}$ be any scalar
            \item also note that $(x^tV)^t=V^tx$ 
            \item thus we can say for an arbitrary $y\in\mathbb{R}^{n}$ we can pick an $x\in \mathbb{R}^n$ such that $x^tV=y^t$ and thus as we know $0\leq x^tAx=x^tVDv^tx=y^tDy$ meaning finally $0\leq y^tDy$
            
        \end{itemize}
        \item now we will show that first we will show if for any $y\in \mathbb{R}^{n} y^TDy\geq0$ then A is positive semi-definite 
        \begin{itemize}
            
            \item we know that for any vector x we will have $x^tDx\geq 0$ so for any vector y$\in \mathbb{R}^d$ we can write $y^tAy=y^tVDV^ty=x^tDx\geq 0$ for some vector $x\in \mathbb{R}^d$ meaning that this will hold for any vector y.  
        \end{itemize}
    \end{enumerate}

    }
            \item Use part (a) to prove that A is positive semidefinite if and only if all the eigenvalues of A are non-negative.
            \red{
        \begin{enumerate}
            \item first we will show that if A is positive semi definite then all eigenvalues are negative. \begin{itemize}
                \item from part A we know that given A is positive semidefininte we will have $y^tDy\geq 0$ for any y$\in \mathbb{R}^d$ 
                \item $0\leq y^tDy=$\begin{pmatrix}
                y1&..&y_n
                \end{pmatrix}\begin{pmatrix}
                \lambda_1 &0&...&0\\
                ...&...&...&...\\
                0&...&...&\lambda_n
                \end{pmatrix}\begin{pmatrix}
                y1\\...\\y_n
                \end{pmatrix}=$\Sigma_{i=1}^{n}y_i^2\lambda_i$
                \item we know that regardless of what vector we chose for y this must hold, so for instance if we chose our Y to be zeros in every entry but one, meaning that it must be the case that all eigenvalues are greater than or equal to zero. 
            \end{itemize}
        \item now consider if all the eigenvalues of A are positive we want to show that A is semidefinite  
        \begin{itemize}
            \item so consider a matrix A with all psotive eigenvalues
            \item consider an arbitary vector v $\in \mathbb{R}^d$ as well as D a diagonal matrix holding the eigenvalues of A
            we can see that \\ $v^tDv=$\begin{pmatrix}
            v_1&..&v_n
            \end{pmatrix}
            \begin{pmatrix}
            \lambda_1 &... &0\\
            0& ... &0\\
            0& ... &\lambda_n\\
            \end{pmatrix}
            \begin{pmatrix}
            v_1\\..\\v_n
            \end{pmatrix}=$\Sigma_{i=1}^{n}v_i^2\lambda_i$ since we know that $v_i^2\geq 0$ is always true and all the eigen values are non-negative it must be the case that $\Sigma_{i=1}^{n}v_i^2\lambda_i\geq 0$ meaning that $v^tDv\geq 0$ for any vector V$\in \mathbb{R}^d$ and thus that A is postive semi defininte
        \end{itemize}
        \end{enumerate}
        }
        \item show that an nxn matrix with enteries of all one is postives semi definite
        \red{
        \begin{itemize}
            \item let V=\begin{pmatrix}
            1&1&...&1\\...&..&...&...\\1&1&...&1
            \end{pmatrix} since all the coloms of V are repreates of eachother we know that they are not linearly depedint and rank(V)=1
            \item thus we know V will have zero as an eigenvalue with multiplcaity zero 
            \item call v=\begin{pmatrix}
            1&...&1
            \end{pmatrix}
            \item now note that V can be expressed as
            $V=\begin{pmatrix}
            1\\...\\1
            \end{pmatrix}\begin{pmatrix}
            1&...&1
            \end{pmatrix}=v^tv$ so thus we can see that $Vv=vv^tv=v||v||^2$ meaning that $||v||^2$ will be an eigen value of V, we can see further that $||v||^2=\Sigma_{i=1}^{n}1^2=\Sigma_{i=1}^{n}1=n$
            \item thus V will have an eigenvalue of n. 
            \item finally we know thhat the sum of the multiplcites of the eigen values of V can at most be n, and we saw that 0 has mutliplicty n-1 and n has mutliplicty at least one. meaning that 0, n are the only possible eigen values whcih are both larger than zero. 
            \item so we know that V has all postive eigenvalues and thus is postive semidefininte
        \end{itemize}
        
        }
        \item leet M$\in \mathbb{R}^{MxN}$ prove that A=$M^TM$ is postive semidefinite 
    \begin{itemize}
    \item for A to be postive semi definite it must be the case that for any vector $x\in\mathbb{R}^n$ $x^tAx\geq0$
    \item so consider $x^tAx=x^tM^tmX=<xM,xM>\geq 0$ regardless of the values of x and m by the properties of inner products.
    \end{itemize}
\end{enumerate}
\end{enumerate}
\end{document}
